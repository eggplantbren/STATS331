\chapter{Rosetta Stone}
Bayesian statistics has a lot of terminology floating around, and sometimes
different communities use different terms for the same concept. This appendix
lists various terms that are are basically synonymous (they mean the same thing).

{\bf Event, Hypothesis, Proposition, Statement}: These are all the same thing.
A statment that can be true or false. These are the things that go in our
probability statements: If we write $P(A|B)$, $A$ and $B$ are both propositions.

{\bf Sampling Distribution, Probability Model for the Data, 
Generative Model, Likelihood Function, Likelihood}: This is the thing that we
write as $p(D|\theta)$. Sometimes it is called the sampling distribution or
a generative model because
you can sometimes think of the data as having been ``sampled from''
$p(D|\theta)$ but using the true value of $\theta$, which you don't actually
know. There is a subtlety here, and that is that the word likelihood can be
used to mean either $p(D|\theta)$ before $D$ is known (in which case it is the
thing you could use to predict possible data) or after $D$ is known (in which
case it gives you the thing you should use to calculate the posterior
probabilities).

{\bf Marginal Likelihood, Evidence, Prior Predictive Probability, Normalising
Constant}: This is the $p(D)$ or $P(D)$ term in the denominator of Bayes' rule,
and it is also the total of the {\tt prior} $\times$ {\tt likelihood} column
of a Bayes' Box.

