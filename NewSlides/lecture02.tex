\documentclass{beamer}
\usetheme[pageofpages=of,% String used between the current page and the
                         % total page count.
          bullet=circle,% Use circles instead of squares for bullets.
          titleline=true,% Show a line below the frame title.
          alternativetitlepage=true,% Use the fancy title page.
       %   titlepagelogo=logo-polito,% Logo for the first page.
       %   watermark=watermark-polito,% Watermark used in every page.
       %   watermarkheight=100px,% Height of the watermark.
       %   watermarkheightmult=4,% The watermark image is 4 times bigger
                                % than watermarkheight.
          ]{Torino}

\setbeamertemplate{footline}{
  \begin{beamercolorbox}[wd=\paperwidth,ht=1ex,dp=1ex]{footline}
    \vspace{5pt} \hspace{1em} \insertframenumber/\inserttotalframenumber
  \end{beamercolorbox}
}

\author{Brendon J. Brewer}
\title{STATS 331 -- Introduction to Bayesian Statistics}
\institute{The University of Auckland}
\date{}


\linespread{1.3}
\usepackage{minted}
\usepackage[utf8]{inputenc}
\usepackage{dsfont}
\newcommand{\given}{\,|\,}


\begin{document}

\frame{\titlepage}

\begin{frame}
\begin{center}
\includegraphics[width=0.3\textwidth]{images/drinking_game.png}
\end{center}

Do not attempt at home.

\end{frame}

\begin{frame}
\frametitle{Maths, Probability, and R}

\begin{itemize}
\item To understand and use Bayesian statistics, we will
require some mathematics and some R programming
skills.\pause
\item  In this lecture we will briefly review some of the concepts
that we will need.\pause
\item If you are a bit rusty, there will be plenty of opportunity to
brush up. If you are already a pro, great!
\end{itemize}

\end{frame}


\begin{frame}
\frametitle{Probability}
We will use probability extensively in this course. You may have previous
experience with it (e.g. from STATS 125), which may help. However, the way we
use it is different.



\end{frame}


\begin{frame}
\frametitle{Probability}
We will use probability extensively in this course. You may have previous
experience with it (e.g. from STATS 125), which may help. However, the way we
use it is different.



\end{frame}

\begin{frame}
\frametitle{Probability --- Product Rule --- Traditional}
If we have two {\color{red} events} $A$ and $B$, the probability that both
{\color{red} occur} is given by the product rule:

\begin{align}
P(A \cap B) &= P(A)P(B \given A)
\end{align}

The `$\cap$' means intersection, but we will not see it again.



\end{frame}

\begin{frame}
\frametitle{Probability --- Product Rule --- Bayesian}
If we have two {\color{blue} propositions} or
{\color{blue} statements}
 $A$ and $B$, the probability that both {\color{blue} are true}
is given by the product rule:

\begin{align}
P(A, B) &= P(A)P(B \given A) \\
        &= P(B)P(A \given B).
\end{align}\pause

Note the change in terminology, and the notation. The comma means `and',
and the vertical bar means `given'.



\end{frame}


\begin{frame}
\frametitle{Probability --- Bayesian}
\begin{itemize}
\item In 331 we apply probability to {\color{blue} propositions
or statements}, not events. \pause
\item Probabilities represent the fact that we don't know everything.
You can think of it as `plausibility'. \pause
\item There is not a concept of `randomness' in the
sense of variability.\pause
\item If a quantity varies (e.g., the maximum daily temperature $T$),
really there’s more than one quantity
$(T_{\rm yesterday}$, $T_{\rm today}, T_{\rm tomorrow}, ...)$. We may or may
not know their values.
\end{itemize}

\end{frame}

\begin{frame}
\frametitle{The Most General Product Rule}
The product rule also holds if every term has `given $C$' on the right hand
side, where $C$ is any third statement:

\begin{align}
P(A, B \given C) &= P(A \given C)P(B \given A, C).
\end{align}\pause

If this is a bit much at this point, don't worry --- we have ways of making it
seem easy.

\end{frame}


\begin{frame}
\frametitle{Product Rule Example}
Consider a particular individual person, who you know very little about.

\begin{itemize}
\item Suppose the probability the person is male is 50\%.
\item Suppose the probability that a male is taller than 6 feet is
30\%.
\item What is the probability that the person is both male and
taller than 6 feet?
\end{itemize}

\end{frame}

\begin{frame}
\frametitle{Product Rule Example}
By the product rule,

\begin{align}
P(\textnormal{male}, \textnormal{tall})
    &= P(\textnormal{male})P(\textnormal{tall} \given \textnormal{male})\\
    &= 0.5 \times 0.3\\
    &= 0.15.
\end{align}

\end{frame}

\begin{frame}
\frametitle{Product Rule --- Probability Interpretation}
Notice that the story was about a {\em single individual} whose properties
were {\em unknown to you}. This is Bayesian.\\[0.5em]\pause

A similar question could ask for the
{\em proportion of the population} that is male and tall.
The mathematics would be exactly the same, but the meaning of the quantities
is different.
\end{frame}

\begin{frame}
\frametitle{Probability --- Sum Rule}
If we have two propositions or
statements
 $A$ and $B$, the probability that either $A$ or $B$ is true (or both)
is given by the sum rule:
\begin{align}
P(A \vee B) &= P(A) + P(B) - P(A, B)
\end{align}\pause

The $\vee$ means `or'. To remember: if a question involves
`and', use the product rule. If it involves `or', use the sum rule.

\end{frame}

\begin{frame}
\frametitle{Probability --- Sum Rule --- Special Case}
We often deal with {\bf mutually exclusive} statements or propositions ---
both can't be true. E.g., `the balance in my bank account is $\$100.00$'
and `the balance in my bank account is $\$105.00$'.
In this case, the final term in the sum rule is equal to zero:
\begin{align}
P(A \vee B) &= P(A) + P(B)
\end{align}\pause

We use this frequently.


\end{frame}


\begin{frame}
\frametitle{The Most General Sum Rule}
We can write down the sum rule in its most general form by inserting
`given $C$' in the right hand side of every term, where $C$ is some third
proposition:

\begin{align}
P(A \vee B \given C) &= P(A \given C) + P(B \given C) - P(A, B \given C).
\end{align}

\end{frame}

\begin{frame}
\frametitle{Partition Theorem and Bayes' Theorem}
There are some other results, called the Partition Theorem
(you may have met this
in STATS 210) and Bayes' theorem. They are derived from the sum and product
rules, and we will meet them soon enough.\\[0.5em]\pause

As you can probably guess, we will use Bayes' theorem a lot.

\end{frame}



\begin{frame}
\frametitle{Random Variables}
We will need to use and understand `random variables'. This is awkward
terminology for a Bayesian, so we won't often say this. But basically
we are talking about {\em quantities that have probability distributions}.

\end{frame}


\begin{frame}
\frametitle{Discrete Probability Distributions: Example}

\begin{center}
\includegraphics[width=0.5\textwidth]{images/geometric.pdf}
\end{center}

We can use the sum rule to calculate things like $P(x > 3)$ which would be
the sum of a subset of the probabilities.

\end{frame}


\begin{frame}
\frametitle{Continuous Probability Distributions: Example}

\begin{center}
\includegraphics[width=0.5\textwidth]{images/normal.pdf}
\end{center}

We can still calculate things like $P(x > 3)$ which would be
given by a definite integral of the probability density function.

\end{frame}

\end{document}

