\documentclass{beamer}
\usetheme[pageofpages=of,% String used between the current page and the
                         % total page count.
          bullet=circle,% Use circles instead of squares for bullets.
          titleline=true,% Show a line below the frame title.
          alternativetitlepage=true,% Use the fancy title page.
       %   titlepagelogo=logo-polito,% Logo for the first page.
       %   watermark=watermark-polito,% Watermark used in every page.
       %   watermarkheight=100px,% Height of the watermark.
       %   watermarkheightmult=4,% The watermark image is 4 times bigger
                                % than watermarkheight.
          ]{Torino}

\setbeamertemplate{footline}{
  \begin{beamercolorbox}[wd=\paperwidth,ht=1ex,dp=1ex]{footline}
    \vspace{5pt} \hspace{1em} \insertframenumber/\inserttotalframenumber
  \end{beamercolorbox}
}

\author{Brendon J. Brewer}
\title{STATS 331 -- Introduction to Bayesian Statistics}
\institute{The University of Auckland}
\date{}


\linespread{1.3}
\usepackage{minted}
\usepackage[utf8]{inputenc}
\usepackage{dsfont}
\newcommand{\given}{\,|\,}

\begin{document}

\frame{\titlepage}

\begin{frame}
\begin{center}
\includegraphics[width=0.9\textwidth]{images/history.png} \\
Credit: Unknown but maybe Bill Jeffreys?
\end{center}

\end{frame}


\begin{frame}

\begin{center}
Summarising Posterior Distributions
\end{center}

\end{frame}

\begin{frame}
\frametitle{The Posterior}
\begin{itemize}
\item In Bayesian statistics, posterior distributions form the complete answer
to the question ``what is known about $\theta$''? \pause
\item From the posterior distribution, you can calculate the probability of
anything else you might be interested in (e.g., predictions).
\end{itemize}

\end{frame}

\begin{frame}
\frametitle{Posterior Summaries}

\begin{itemize}
\item The posterior might be a complicated probability distribution, or it
might be simple, but in any case it is nice to be able to communicate what it
looks like to others. \pause
\item If we summarise with a `point estimate' (single number guess) or
an interval, then we have some things which are analogous to frequentist
estimators and confidence intervals.\pause
\item Posterior summaries are often analogous to simple summaries of datasets.
\end{itemize}

\end{frame}


\begin{frame}
\frametitle{Simple Summaries}
Here are some summaries which are common, and you might have already thought
of them. We often want to convey something about where the distribution is
centered and how wide it is.\pause

The posterior mean (expected value) and standard deviation are reasonable
choices for this in many cases.\pause

{\bf posterior mean $\pm$ posterior s.d.}
is approximately a 68\% credible interval
(we will see these later) if the posterior is approximately a normal/gaussian
distribution.

\end{frame}

\begin{frame}[fragile]
\frametitle{Computing Simple Summaries}
\begin{minted}{r}
# Posterior mean
post_mean = sum(theta*post)

# Variance and s.d.
post_var = sum((theta - post_mean)^2*post)
post_sd = sqrt(post_var)
\end{minted}

\end{frame}


\begin{frame}[fragile]
\frametitle{Example Posterior Distribution}
Remember our election polling example, with 6 successes out of 10 Bernoulli
trials. The posterior was a Beta$(7, 5)$ distribution.

\begin{center}
\includegraphics[width=0.65\textwidth]{images/beta_posterior.pdf}
\end{center}

\end{frame}


\begin{frame}[fragile]
\frametitle{Example Posterior Distribution}

The mean of this is 7/12 $\approx 0.583$ (as we saw when we did prediction)
and the s.d. is available analytically. We get
\begin{align}
\theta = 0.583 \pm 0.137.
\end{align}


\end{frame}


\begin{frame}
\frametitle{Other Options for Point Estimates}
The posterior mean is not the only option for presenting a point estimate.
We also have, for example, the posterior median or the posterior mode.\\[0.5em]\pause

Is there any way we can say one of these is a better choice than another?

\end{frame}

\end{document}

