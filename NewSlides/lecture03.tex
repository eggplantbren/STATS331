\documentclass{beamer}
\usetheme[pageofpages=of,% String used between the current page and the
                         % total page count.
          bullet=circle,% Use circles instead of squares for bullets.
          titleline=true,% Show a line below the frame title.
          alternativetitlepage=true,% Use the fancy title page.
       %   titlepagelogo=logo-polito,% Logo for the first page.
       %   watermark=watermark-polito,% Watermark used in every page.
       %   watermarkheight=100px,% Height of the watermark.
       %   watermarkheightmult=4,% The watermark image is 4 times bigger
                                % than watermarkheight.
          ]{Torino}

\setbeamertemplate{footline}{
  \begin{beamercolorbox}[wd=\paperwidth,ht=1ex,dp=1ex]{footline}
    \vspace{5pt} \hspace{1em} \insertframenumber/\inserttotalframenumber
  \end{beamercolorbox}
}

\author{Brendon J. Brewer}
\title{STATS 331 -- Introduction to Bayesian Statistics}
\institute{The University of Auckland}
\date{}


\linespread{1.3}
\usepackage{minted}
\usepackage[utf8]{inputenc}
\usepackage{dsfont}
\newcommand{\given}{\,|\,}


\begin{document}

\frame{\titlepage}

\begin{frame}
\frametitle{Probability as Plausibility}
The main distinction between Bayesian and frequentist/classical statistics
is the meaning of probability. What we think a probability {\em is}
influences {\em what calculations we think it is valid to do.}\\[0.5em]\pause

In Bayesian statistics, $P(A \given B)$ means how plausible statement $A$ is,
assuming that $B$ is true (this might be hypothetical --- {\em if} $B$
were true --- or we might actually know $B$ is true).

\end{frame}



\begin{frame}
\frametitle{Example of Probability as Plausibility}
Suppose there is some {\bf statement}
or {\bf proposition} $A$, and we are not sure whether it is true or
false. For example, {\em $A \equiv$ New Zealand wins more bronze medals than
Australia at the 2028 Olympics}.\pause

We can describe the degree of plausibility of $A$ using a value between 0 and 1.
For instance:
\begin{itemize}
\item $P(A) = 1$ (it is definitely true) \\
\item $P(A) = 1$ (it is definitely false) \\
\item $P(A) = 0.1$ (it is probably false) \\
\item $P(A) = 0.5$ (as certain as a coin landing heads)
\end{itemize}

\end{frame}


\begin{frame}
\frametitle{Bayesian Interpretations of Probability}
There are some language choices people use to describe what
Bayesian probabilities represent. Examples:\pause

\begin{itemize}
\item Plausibility \pause
\item Degree of confidence \pause
\item Degree of belief \pause
\item The degree to which one statement implies another
\end{itemize}

\end{frame}



\begin{frame}
\frametitle{Subjective Probabilities}
Some people say that Bayesian probabilities are subjective.
This is often true. It is certainly the case
that they depend on what information you have.\\[0.3em]\pause

Consider the statement that NZ's population is greater than 8
million people. I would assign a probability close to zero.
Someone from afar might assign a probability closer to 0.5.\\[0.3em]\pause

When different people have different probability judgments,
it is usually because they have different information or assumptions.

\end{frame}


\begin{frame}
\frametitle{Assumptions}
\begin{itemize}
\item Even though different people might have different probability
judgments, the rules linking one probability to another --- the sum
and product rule (etc) --- are consistent.\\\pause
\item When we start working on problems, we will see that some
input probabilities in the problem
are judgment calls, but other ones are calculated.\\\pause
\item The most obvious place where assumptions are needed is the
`prior' which we will meet soon, but it's not the only place
where assumptions enter.
\end{itemize}

\end{frame}


\begin{frame}
\frametitle{Updating Probabilities}
There is a well-defined procedure for updating probabilities to take new
information into account. For example, suppose you are interested in
the proposition $A$.\pause
\begin{itemize}
\item You assign a {\bf prior} probability $P(A)$.\pause
\item Then, you learn that
some related proposition $B$ is true. You should update your probability
to $P(A \given B)$, called a {\bf posterior} probability.\pause
\item If you then learn that $C$ is true, you should update
to $P(A \given B, C)$, another posterior probability.\pause
\item And so on.
\end{itemize}

\end{frame}



\begin{frame}
\frametitle{Updating Probabilities: Bayes' Rule}
Bayes' rule (derived from the product rule) gives us the procedure to
calculate updated (posterior) probabilities.

\begin{align}
P(H \given D) &= \frac{P(H) P(D\given H)}{P(D)}.
\end{align}

\pause
The terms have names:
\begin{align}
\textnormal{posterior prob.} &=
        \frac{\textnormal{prior prob.} \times \textnormal{likelihood}}{\textnormal{marginal likelihood}}
\end{align}

\end{frame}



\begin{frame}
\frametitle{Updating Probabilities: Bayes' Rule}
The statements in the previous equation were $H$ and $D$.
These stand for {\em hypothesis} and {\em data} respectively,
to indicate how we use Bayes' theorem.\\[0.5em]\pause

We usually want to know if $H$ is true, and we don't know,
so we calculate a probability for it. We usually have some
data proposition $D$ which we do know is true, so we use that.

\end{frame}


\begin{frame}
\frametitle{Interpretation of the Terms}

\begin{itemize}
\item $P(H\given D)$ is the updated (posterior) probability, which we
want.\pause
\item $P(H)$ is the initial (prior) probability. It describes
how plausible $H$ is without taking $D$ into account.\pause
\item $P(D \given H)$ is the likelihood value --- how plausible would $D$
have been if we knew $H$ was true?\pause
\item $P(D)$ is the marginal likelihood\footnote{Sometimes called the
{\bf evidence}, especially by physicists.} --- how plausible would $D$ have been
whether $H$ is true or not?
\end{itemize}

\end{frame}


\begin{frame}
\frametitle{Input and Output}
Two of the probabilities --- the prior and the likelihood --- are inputs to
the calculation, that we assign based on judgment calls or prior/external
information.\\

The other two --- the posterior and the marginal likelihood --- are outputs
that we compute.
\end{frame}



\end{document}

