\chapter{Markov Chain Monte Carlo}

\section{Monte Carlo}
Monte Carlo is a computational technique for handling probability distributions.
What we have seen so far is that you can represent a probability distribution
in a computer by using a vector of possible values and a corresponding
vector of probabilities:
\begin{verbatim}
theta <- seq(0, 1, by=0.1)
prior <- rep(1, length(theta))/length(theta)
\end{verbatim}
Then, if you want to plot the probability distribution you can:
\begin{verbatim}
plot(theta, prior, xlab='Theta', ylab='Prior Probability')
\end{verbatim}
If you want to get some summaries, say, the prior mean and standard deviation,
you could also compute those quite easily:
\begin{verbatim}
prior_mean <- sum(theta*prior)
prior_sd <- sqrt(sum(theta**2*prior) - prior_mean**2))
\end{verbatim}
However, once life starts to get more complicated, such things will no longer
be possible. This is especially true when we work with situations where there
is {\it more than one} unknown parameter. Say we have five unknown parameters
$\theta_1$, $\theta_2$, $\theta_3$, $\theta_4$ and $\theta_5$. Then one
possibility for the values of the parameters might be
$\{\theta_1, \theta_2, \theta_3, \theta_4, \theta_5\} =
\{0.1, 3.5, 2.3, 0.8, -4.3\}$. Another possibility is
$\{\theta_1, \theta_2, \theta_3, \theta_4, \theta_5\} =
\{0.4, 0, -0.3, 10.9, -2.3\}$. Can you see what the problem is here?
The vector of possible values you would need becomes very large very
quickly.




\section{The Metropolis Algorithm}

\section{Tactile MCMC}

Tactile MCMC is an idea due to Wayne Stewart.
