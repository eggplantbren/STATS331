\chapter{Probability and Uncertainty}

Every day, throughout our lives, we are required 
to believe certain things and not to believe other things. I'm not just 
referring to the ``big questions'', but also to trivial matters, and 
everything in between. For example, this morning I boarded the bus to 
campus, sure that it would actually take me there and not to Las Vegas. 
Of course, I did not consciously consider the hypothesis that the bus 
would take a dramatic detour. However, had I been prompted to, I would 
have stated that it was at least implausible - assuming, of course, that 
being asked this strange question wouldn't raise my suspicions.

Somehow, our brains are able to accurately predict the correct answer
(e.g. the destination of a bus), without much conscious attention or
problem solving. However, there are areas of study where we are unable
to make such accurate judgments intuitively. Most of science involves
such situations - those questions that have already been solved are
no longer of interest.
