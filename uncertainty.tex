\chapter{Introduction}
Every day, throughout our lives, we are required 
to believe certain things and not to believe other things. This applies not
only to the ``big questions'' of life, but also to trivial matters, and 
everything in between. For example, this morning I boarded the bus to 
university, sure that it would actually take me here and not to Wellington.
How did I know the bus would not take me to Wellington? Well, for starters
I have taken the same bus many times before and it has always taken me to the
university. Another clue was that the bus said ``Midtown'' on it, and a bus
to Wellington would probably have said Wellington, and would not have stopped
at a minor bus stop in suburban Auckland.
None of this evidence {\it proves} that the bus would take me to university,
but it does makes it plausible. Given all these pieces of information, I feel
quite certain that the bus will take me to the city. I feel so certain
about this that the possibility of an
unplanned trip to Wellington never even entered my mind until I decided to
write this paragraph.

Somehow, our brains are very often able to accurately predict the correct answer
to many questions (e.g. the destination of a bus), even though we don't have
all of the available information that we would need to be 100\% certain.
We do this using our experience of the world and our intuition, usually 
without much conscious attention or problem solving. However, there are areas
of study where we can't just use our intuition to make judgments like this.
For example, most of science involves such situations - people tend to be
interested in trying to answer questions that haven't yet been answered!
This is where statistics comes in: to help us in this grey area where we can't
be 100\% certain about what's true.


\subsection{Certainty, Uncertainty and Probability}


